\chapter{ตัวอย่างภาษาไทย}
\begin{summary}
 ในบทนี้จะเป็นตัวอย่างการใช้ภาษาไทยร่วมกับภาษาอังกฤษ  โดยทำการแปลงจากเนื้อหาในบทที่ 1 เป็นภาษาไทย การใช้งานเป็นอย่างไร มาลองดูกัน  ดูตัวอย่างได้จาก code ที่ใช้ทำเอกสารนี้ได้ลย 
\end{summary}

\section{การเขียนทฤษฎี}

ลองเริ่มต้นด้วยการเขียนเกี่ยวกับทฤษฏี   จะเห็นว่าหัวข้อ theorem ยังคงเป็นภาษาอังกฤษอยู่ ถ้าต้องการเปลี่ยนไปเป็นภาษาไทยทำได้โดยการไปแก้ไขใน class หรืออาจจะทำเป็นรูปแบบของการแปลภาษาก็น่าจะสะดวกไว้มีเวลาจะดำเนินการต่อ ตอนนี้ขอเป็นแบบ ภาษาอังกฤษไปก่อน
\begin{theorem}[Logic algebra]
  \label{th:logicalgebra}
  \index{logic algebra}
  กำหนดให้ $P$, $Q$ และ $R$  มีทฤษฎีบทเป็นลอจิกคือ จริง หรือ เท็จ ดังนั้นทฤษฎีบทต่อไปนี้จะเป็นจริง
  \small
  \begin{align*}
    P \land Q &\Leftrightarrow Q \land P &
    P \lor  Q &\Leftrightarrow Q \lor P  &&
    \text{(commutative laws)} \\
    (P \land Q) \land R &\Leftrightarrow P \land (Q \land R) &
    (P \lor Q)  \lor  R &\Leftrightarrow P \lor  (Q \lor  R) &&
    \text{(associative laws)} \\
    P \land (Q \lor  R) &\Leftrightarrow (P \land Q) \lor  (P \land R) &
    P \lor  (Q \land R) &\Leftrightarrow (P \lor  Q) \land (P \lor  R) &&
    \text{(distributive laws)} \\
    \lnot (P \land Q) &\Leftrightarrow \lnot P \lor  \lnot Q &
    \lnot (P \lor  Q) &\Leftrightarrow \lnot P \land \lnot Q &&
    \text{(De Morgan's laws)}
  \end{align*}
\end{theorem}
\begin{proof}
  \newcommand{\T}{\mathsf{T}}
  \newcommand{\TT}{\mathbf{T}}
  \renewcommand{\F}{\mathsf{F}}
    เราพิสูจน์กฎแรกของเดอร์แกนและทิ้งส่วนที่เหลือไว้เป็นแบบฝึกหัด 
    เพื่อพิสูจน์กฎดังกล่าว เราสร้างตารางความจริงและใส่ค่าที่เป็นไปได้ทั้งหมดซึ่งประกอบไปด้วย จริง หรือ เท็จ ของ 
   $ P $ และ $ Q $   หลังจากนั้นเราได้ตารางความจริงซึ่งตารางจะประกอบไปด้วย 4 กรณี ดังนี้
  \begin{center}
    \begin{tabular}{cccccccccc}
      $\lnot$ & ($P$ & $\land$ & $Q$) & $\Leftrightarrow$ & $\lnot$ & $P$ & $\lor$ & $\lnot$ & $Q$ \\
      \midrule
      & $\T$ && $\T$ &&& $\T$ &&& $\T$ \\
      & $\T$ && $\F$ &&& $\T$ &&& $\F$ \\
      & $\F$ && $\T$ &&& $\F$ &&& $\T$ \\
      & $\F$ && $\F$ &&& $\F$ &&& $\F$
    \end{tabular}
  \end{center}
    จากนิยามของการดำเนินการทางด้านลอจิก เราสามารถสร้างตารางที่สมบูรณ์ได้ดังนี้ 
  \begin{center}
    \begin{tabular}{cccccccccc}
      $\lnot$ & ($P$ & $\land$ & $Q$) & $\Leftrightarrow$ & $\lnot$ & $P$ & $\lor$ & $\lnot$ & $Q$ \\
      \midrule
      $\F$ & $\T$ & $\T$ & $\T$ & $\TT$ & $\F$ & $\T$ & $\F$ & $\F$& $\T$ \\
      $\T$ & $\T$ & $\F$ & $\F$ & $\TT$ & $\F$ & $\T$ & $\T$ & $\T$& $\F$ \\
      $\T$ & $\F$ & $\F$ & $\T$ & $\TT$ & $\T$ & $\F$ & $\T$ & $\F$& $\T$ \\
      $\T$ & $\F$ & $\F$ & $\F$ & $\TT$ & $\T$ & $\F$ & $\T$ & $\T$& $\F$
    \end{tabular}
  \end{center}
  ตามนิยามที่ได้กำหนดไว้ตอนต้นเราสามารถพิสูจน์ให้เห็นว่าค่าประพจน์ที่กำหนดนั้นเป็นจริง (the equivalence $\Leftrightarrow$) ทุกเงื่อนไขที่เป็นไปได้ 
 \end{proof}

\section{การนิยาม}

ในเนื้อหาส่วนนี้เราจะพูดถึงการนิยาม หรือ definition กัน ว่าใน class นี้จะออกมาหน้าตาอย่างไร และใช้งานอย่างไรให้ดู code ประกอบไปด้วย 

\begin{definition}[Rational Cauchy sequence]
  \label{th:rationalcauchysequence}
  \index{rational Cauchy sequence}
  A rational Cauchy sequence is a rational sequence
  $(x_n)_{n=0}^{\infty}$ such that
  \begin{equation}
    \forall \epsilon \in \mathbb{Q}_+ \;
    \exists N \in \mathbb{N} : m, n \geq N \Rightarrow |x_m - x_n| < \epsilon.
  \end{equation}
  In other words, for each (small) rational number $\epsilon > 0$
  there is a (big) number $N$ such that the distance $|x_m - x_n|$
  between $x_m$ and $x_n$ is less than $\epsilon$ if both $m$ and $n$
  are larger than or equal to $N$.
\end{definition}

\begin{remark}
 การเขียนหมายเลขของสมการ ก็ใช้แบบเดี่ยวกับเขียนในสมการทั่วไป
 \end{remark}

\begin{example}[แก้สมการ $x^2 = 2$]
  Consider the equation $x^2 = 2$. It is easy to prove that this
  equation does not have any rational solutions. However, consider
  the following iteration formula:
  \begin{equation}
    x_n = \frac{x_{n-1} + 2 / x_{n - 1}}{2},
  \end{equation}
  where $n = 1,2,3,\ldots$ and $x_0 = 1$. The resulting sequence of
  rational numbers quickly approaches a number in the vicinity of
  $x = 1.4142135623731$:
  \begin{displaymath}
    \begin{array}{rclcl}
      x_0 &=& 1 \\
      x_{1} &=& (x_{0} + 2 / x_{0}) / 2 &=& 1.5 \\
      x_{2} &=& (x_{1} + 2 / x_{1}) / 2 &\approx& 1.4166666666667 \\
      x_{3} &=& (x_{2} + 2 / x_{2}) / 2 &\approx& 1.4142156862745 \\
      x_{4} &=& (x_{3} + 2 / x_{3}) / 2 &\approx& 1.4142135623747 \\
      x_{5} &=& (x_{4} + 2 / x_{4}) / 2 &\approx& 1.4142135623731 \\
      x_{6} &=& (x_{5} + 2 / x_{5}) / 2 &\approx& 1.4142135623731 \\
      x_{7} &=& (x_{6} + 2 / x_{6}) / 2 &\approx& 1.4142135623731 \\
      x_{8} &=& (x_{7} + 2 / x_{7}) / 2 &\approx& 1.4142135623731 \\
      x_{9} &=& (x_{8} + 2 / x_{8}) / 2 &\approx& 1.4142135623731 \\
      x_{10} &=& (x_{9} + 2 / x_{9}) / 2 &\approx& 1.4142135623731
    \end{array}
  \end{displaymath}
  We will later see that this iteration, or any other equivalent
  iteration, defines the real number $\sqrt{2}$.
\end{example}

\section{การใส่ Code }
 เนื่องจากหนังสือที่เขียนจะต้องมีส่วนประกอบของ code ด้วย ดังนั้นส่วนนี้จึงจะต้องทำเพิ่มเติมขึ้นมากจากที่ code เดิมได้เตรียมไว้ให้ ผลลัพธ์จะออกมาเป็นอย่างไรมาลองดูกันเลย 
สำหรับวิธีใช้ไปดูจากตัว source code กันนะ ไม่ขออธิบายในนี้ 

\begin{lstlisting}[language=python, caption=Python example]
import numpy as np
    
def incmatrix(genl1,genl2):
    m = len(genl1)
    n = len(genl2)
    M = None #to become the incidence matrix
    VT = np.zeros((n*m,1), int)  #dummy variable
    
    #compute the bitwise xor matrix
    M1 = bitxormatrix(genl1)
    M2 = np.triu(bitxormatrix(genl2),1) 

    for i in range(m-1):
        for j in range(i+1, m):
            [r,c] = np.where(M2 == M1[i,j])
            for k in range(len(r)):
                VT[(i)*n + r[k]] = 1;
                VT[(i)*n + c[k]] = 1;
                VT[(j)*n + r[k]] = 1;
                VT[(j)*n + c[k]] = 1;
                
                if M is None:
                    M = np.copy(VT)
                else:
                    M = np.concatenate((M, VT), 1)
                
                VT = np.zeros((n*m,1), int)
    
    return M
\end{lstlisting}

\section*{Exercises}

\begin{exercise}
  Let $A = \{1, 2, 3\}$ and $B = \{2, 3, 4\}$.
  Determine the following sets. \\
  (a) $A \cup B$ \quad
  (b) $A \cap B$ \quad
  (c) $A \setminus B$ \quad
  (d) $A \times B$
\end{exercise}

\begin{exercise}
  Let $A = \{1, 3, 5, 7, 9\}$ and $B = \{2, 4, 6, 8, 10\}$.
  Determine the following sets. \\
  (a) $A \cup B$ \quad
  (b) $A \cap B$ \quad
  (c) $A \setminus B$ \quad
  (d) $A \times B$
\end{exercise}

\begin{exercise}
  Let $A = \{1, 2, 3\}$, $B = \{2, 3, 4\}$ and $C = \{3, 4, 5\}$.
  Determine the following sets. \\
  (a) $A \cup B \cup C$ \quad
  (b) $A \cap B \cap C$ \quad
  (c) $(B \setminus A) \cap C$ \quad
  (d) $(A \times B) \times C$
\end{exercise}

\section*{Problem}

\begin{problem}
  Interpret the following set definition (Russell's paradox) and discuss
  whether $X \in X$ or $X \notin X$:
  \begin{equation}
    X = \{x \mid x \notin x\}.
  \end{equation}
\end{problem}

\section*{Computer exercises}

\begin{programming}
  Write a program that generates the sequence $(x_n)_{n=0}^{100}$
  for $x_n = n$.
\end{programming}

\begin{programming}
  Write a program that generates the odd numbers between $1$ and $100$.
\end{programming}

\begin{programming}
  Write a program that computes the sum $\sum_{n=0}^{100} x_n$
  for $x_n = n$.
\end{programming}